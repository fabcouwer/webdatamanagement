\documentclass[11pt]{article}
\usepackage[english]{babel}
\usepackage[utf8x]{inputenc}
\usepackage{fullpage}
\usepackage{color}
\usepackage{caption}
\usepackage{todonotes}
\usepackage{listings}
\usepackage{courier}
\usepackage{float}
\usepackage{hyperref}
\usepackage[]{natbib}

%eclipse colors
\definecolor{sh_comment}{rgb}{0.12, 0.38, 0.18}
\definecolor{sh_keyword}{rgb}{0.37, 0.08, 0.25}
\definecolor{sh_string}{rgb}{0.06, 0.10, 0.98}

\lstdefinelanguage{PIG}{
    morecomment=[l]{--},
    morekeywords={
        LOAD,
        FOREACH,
        GENERATE,
        GROUP,
        COGROUP,
        AS,
        BY,
        DUMP,
        STORE,
        INTO
        },
    morestring=[b]',
    sensitive=false
}

\lstset{
    language=Java,
    basicstyle=\footnotesize\ttfamily,
    numbers=left,
    numberstyle=\tiny,
    stepnumber=1,
    numbersep=5pt,
    tabsize=2,
    extendedchars=true,
    breaklines=true,
	frame=b,
    stringstyle=\color{sh_string},
    keywordstyle=\color{sh_keyword}\bfseries,
    commentstyle=\color{sh_comment}\itshape,
    showspaces=false,
    showtabs=false,
    xleftmargin=17pt,
    framexleftmargin=17pt,
    framexrightmargin=5pt,
    framexbottommargin=4pt,
    showstringspaces=false
 }
 \lstloadlanguages{
         XML,
         PIG,
         Java
 }

\DeclareCaptionFont{white}{\color{white}}
\DeclareCaptionFormat{listing}{\colorbox[cmyk]{0.43, 0.35, 0.35,0.01}{\parbox{\textwidth}{\hspace{15pt}#1#2#3}}}
\captionsetup[lstlisting]{format=listing,labelfont=white,textfont=white, singlelinecheck=false, margin=0pt, font={bf,footnotesize}}


\title{Large-Scale Data Management with Hadoop}
\author{Friso Abcouwer - 4019873 \and Matthijs van Dorth - 1265911}

\begin{document}

\maketitle

\section{Introduction}
In this report, we will present our solutions to the exercises in Chapter 19 of the Web Data Management book.
Our code can be found in the /src folder in the zip-file we handed in, and our input and output files can be found in the /input and /output folders, respectively.

\section{Combiner Functions}
After implementing the example MapReduce job, making the Combiner was as simple as copying the Reducer and setting it in the Job class. (Looking back, we could also have simply set the Reducer as the Combiner.)
Without the Combiner, the Mapper will send pairs of the form $<author, 1>$ to the Reducer. However, the Combiner intercepts these and instead sends pairs of the form $<author,N>$ to the Reducer, reducing the disk space needed by the Reducer because the intermediate result has already been generated.

\section{Movies}

\section{PIGLatin Scripts}
The scripts we used can be found in Listings~\ref{lst:pig-ex1} through~\ref{lst:pig-ex5}.
\lstinputlisting[language=PIG, label=lst:pig-ex1, caption=Pig exercise 1, float=h]{pig-exercise1.txt}
\lstinputlisting[language=PIG, label=lst:pig-ex2, caption=Pig exercise 2, float=h]{pig-exercise2.txt}
\lstinputlisting[language=PIG, label=lst:pig-ex3, caption=Pig exercise 3, float=h]{pig-exercise3.txt}
\lstinputlisting[language=PIG, label=lst:pig-ex4, caption=Pig exercise 4, float=h]{pig-exercise4.txt}
\lstinputlisting[language=PIG, label=lst:pig-ex5, caption=Pig exercise 5, float=h]{pig-exercise5.txt}

\section{Inverted File}


\end{document}
