\documentclass[12pt]{article}
\usepackage[english]{babel}
\usepackage[utf8x]{inputenc}
\usepackage{fullpage}

\usepackage{listings}
\usepackage{color}
\usepackage{courier}
\usepackage{caption}
\usepackage{todonotes}

\title{Tree Pattern Evaluation using SAX}
\author{Friso Abcouwer \and Matthijs van Dorth}

\definecolor{mygreen}{rgb}{0,0.6,0}
\definecolor{mygray}{rgb}{0.5,0.5,0.5}
\definecolor{mymauve}{rgb}{0.58,0,0.82}
  
\lstset{
       basicstyle=\footnotesize\ttfamily,
       numberstyle=\tiny,
       stepnumber=1,
       numbersep=5pt,
       tabsize=2,
       extendedchars=true,
       breaklines=true,
       keywordstyle=\color{red},
       frame=b,         
       stringstyle=\color{white}\ttfamily,
       showspaces=false,
       showtabs=false,
       xleftmargin=17pt,
       framexleftmargin=17pt,
       framexrightmargin=5pt,
       framexbottommargin=4pt,
       %backgroundcolor=\color{lightgray},
       showstringspaces=false   
}
\lstloadlanguages{
        Java
}
\DeclareCaptionFont{blue}{\color{blue}} 
\captionsetup[lstlisting]{singlelinecheck=false, labelfont={blue}, textfont={blue}}
  
\DeclareCaptionFont{white}{\color{white}}
\DeclareCaptionFormat{listing}{\colorbox[cmyk]{0.43, 0.35, 0.35,0.01}{\parbox{\textwidth}{\hspace{15pt}#1#2#3}}}
\captionsetup[lstlisting]{format=listing,labelfont=white,textfont=white, singlelinecheck=false, margin=0pt, font={bf,footnotesize}}

\begin{document}

\maketitle

\section{Introduction}
In this report we will show how we implemented an algorithm for evaluating tree-pattern queries using SAX. SAX (Simple API for XML) is a way to parse XML document using a stream of data in contrast to DOM (Document Object Model) which is a complete representation of the XML file as a tree.

We will use the C-TP dialect

\section{The Basic Algorithm}
The basic algorithm 

\todo{explain basic datastructures}
\begin{lstlisting}[language=java, caption=Match.java]
class Match {
    int start, state;
    
}
\end{lstlisting}

\section{The Extended Algorithm}


\end{document}